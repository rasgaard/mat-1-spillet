% ENVIRONMENT DEFINITION
\newcounter{definition}
\newenvironment{definition}[1]{
\definecolor{defcol}{HTML}{2b53a0}
\begin{tcolorbox}[  arc = 0px, 
                    boxrule = 1px, 
                    coltitle = blue,
                    colback = blue!3,
                    colframe = defcol!55]

\refstepcounter{definition}
\large\textbf{\textcolor{defcol}{{\sffamily{1.\thedefinition \quad #1}}}}


\bigskip
}
{
\medskip
\end{tcolorbox}
}


% CENTERPIECE PLACEMENT
\begin{tikzpicture}[remember picture, overlay]
    \node[anchor=north west, 
        xshift=17.5cm, 
        yshift=-16cm] 
        at (current page.north west) 
        {
        
% CENTERPIECE CONTENT
\begin{minipage}{.301\textwidth}
\begin{center}
    {\vspace{2mm}{\Huge \bf Mat 1 spillet} \\ \vspace{2mm}}
\end{center}

\begin{definition}{Skemaer}
\begin{description}
    \item \textbf{A:}
    \item \textbf{B:}
    \item \textbf{C:} Du er på Skema C, så det første du gør er at vælge et af de to andre skemar, som du streamer fremover. Når der er $n$ skema-specifikke tårer til det du streamer, tager du $\frac{n}{2}$ tårer
\end{description}
\end{definition}


\begin{definition}{Hjemmeopgavesæt}
På baggrund af romertallet, som du har fået på dit hjemmeopgavesæt, må du give et antal tårer ud.
\begin{center}
\begin{tabular}{c|c|c|c}
     I & II & III & IV \\ \hline
     10 & 6 & 4 & 2 
\end{tabular}
\end{center}
\end{definition}

    
\end{minipage}
        
}; 
\end{tikzpicture}
% \backgroundsetup{contents=\includegraphics{pictures/michael.png}, scale=1, angle=0, opacity=.7}